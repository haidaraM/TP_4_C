\documentclass[12pt]{article}

\usepackage[utf8]{inputenc}
\usepackage[francais]{babel}
\usepackage[T1]{fontenc}  
\usepackage{fancyhdr}
\usepackage[top=2.5cm,bottom=2.5cm,left=2.5cm,right=2.5cm]{geometry}
\usepackage{lmodern}
\usepackage{listings}
\usepackage{color}
\usepackage{blindtext}
\usepackage[colorlinks=true,urlcolor=black,linkcolor=black]{hyperref}
\definecolor{grey}{rgb}{0.3,0.3,0.3}
\usepackage{graphicx}

\lstset{
language=C++,
basicstyle=\footnotesize\ttfamily,
numberstyle=\normalsize,
numbersep=7pt,
keywordstyle=\color{blue},
commentstyle=\color{grey}
}

\newcommand\Titre{TP4 : H\'eritage, Polymorphisme}
\newcommand\Dater{Pour le 6 f�vrier 2015}
\newcommand{\Numbi}{B3425}
\newcommand{\Membres}{\textsc{Bai} Emilien,
\newline \textsc{Haidara} Mohamed}

\title{\Titre \newline \large Document de conception}
\author{Bin\^ome \Numbi{}: \Membres}
\date{\Dater}

\begin{document}

\pagestyle{fancy}
\renewcommand{\footrulewidth}{1pt}
\renewcommand{\headheight}{1cm}
\renewcommand{\contentsname}{Table des mati�res}



\lhead{\Membres}
\chead{\Titre}
\rhead{\Numbi}

\begin{center}
\begin{LARGE}
\begin{bfseries}

\vspace{1\baselineskip}

\underline{\Titre}
~\newline~\newline \begin{large} \'Evaluation des performances\end{large}
\end{bfseries}
\end{LARGE}
\end{center}
~\\
Nos tests de performance ont \'et\'e r\'ealis\'es sur les machines du d\'epartement, en lecture d'un fichier avec la commande LOAD (cf. tableau \ref{tab:Performances en lecture}) ainsi qu'en \'ecriture d'une sauvegarde (cf figure \ref{tab:Performances en \'ecriture}) et sans traces sur la sortie standard, car elle s'av\`erent tr\`es chronophages. Chacun des ajouts a \'et\'e effectu\'e 10 fois en lecture et 100 fois en \'ecriture afin d'obtenir un temps moyen au plus proche de la r\'ealit\'e. Les temps ont \'et\'e r\'ecup\'er\'es gr\^ace \`a l'usage de traceurs plac\'es au d\'ebut et \`a la fin de l'\'ex\'ecution

\begin{figure}[!h]
\begin{center}
\begin{tabular}{|c|c|}
\hline
~~Nombre d'ajouts de cercles du fichier en lecture (LOAD)~~ & Temps moyen en secondes\\
\hline
1000 & 0,005\\
\hline
10000 & 0,06\\
\hline
100000 & 0,6\\
\hline
500000 & 2,97\\
\hline
1000000 & 6\\
\hline
1500000 & 9,2\\
\hline
2000000 & 12,51\\
\hline
\end{tabular}
\caption{Performances en lecture}
\label{tab:Performances en lecture}
\end{center}
\end{figure}

\begin{figure}[!h]
\begin{center}
\begin{tabular}{|c|c|}
\hline
Nombre de sauvegardes de cercles avec la commande SAVE & Temps moyen en secondes \\
\hline
1000 & 0,0006\\
\hline
10000 & 0,005\\
\hline
100000 & 0,06 \\
\hline
500000 & 0,29\\
\hline 
1000000 & 0,61\\
\hline
1500000 & 0,91\\
\hline
2000000 & 1,19\\
\hline
\end{tabular}
\caption{Performances en \'ecriture}
\label{tab:Performances en \'ecriture}
\end{center}
\end{figure}


\end{document}